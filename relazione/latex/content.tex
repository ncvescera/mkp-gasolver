\section{Obiettivo}
Il problema dello zaino multidimensionale (Multidimensional Knapsack Problem) è
un'estensione del più noto problema dello Zaino. L'obbiettivo è sempre lo
stesso, trovare un set di oggetti che massimizzi il profitto totale facendo in
modo di non superare la capienza massima dello zaino, solo con l'aggiunta di più
vincoli: non dovremmo solo preoccuparci della capienza dello zaino ma anche di
altri $n$ differenti fattori.\\

Questo problema può essere formalmente sintetizzato come segue:

\[
    \max \sum_{j=1}^{n} p_j x_j
\]

\[
    \text{subject to: } \sum_{j=1}^{n} r_{i,j} x_j \leq b_i, \ \ i = 1,2,\ldots, m
\]

con $x_j = \{0, 1\}$, $n$ il numero di oggetti, $m$ il numero di vincoli, $b$ il limite massimo per ogni vincolo,
$r$ il valore per ogni singolo vincolo di ogni oggetto.
Il problema verrà risolto mediante l'implementazione di un Algoritmo Genetico.

\section{Istanze del Problema}

L'algoritmo per la risoluzione di questo problema verrà testato utilizzando il dataset
\href{http://people.brunel.ac.uk/~mastjjb/jeb/orlib/mknapinfo.html}{OR-Library}: una
raccolta di varie istanze, di differenti dimensioni, per una svariata moltitudine di
problemi. Una singola istanza di presenta come segue:

\begin{lstlisting}[caption={Esempio di Istanza di un problema MKP con 6 oggetti e 10 vincoli.}]
6
100,600,1200,2400,500,2000
10
8,12,13,64,22,41
8,12,13,75,22,41
3,6,4,18,6,4
5,10,8,32,6,12
5,13,8,42,6,20
5,13,8,48,6,20
0,0,0,0,8,0
3,0,4,0,8,0
3,2,4,0,8,4
3,2,4,8,8,4
80,96,20,36,44,48,10,18,22,24
3800
\end{lstlisting}

La prima riga contiene il numero di oggetti, la seconda il valore di ogni singolo
oggetto, la terza il numero di parametri, le successive righe rappresentano i
valori dei coefficienti del primo parametro, del secondo e così via fin quando non si raggiunge il
numero di parametri. La penultima riga indica il valore massimo per ogni parametro (rappresenta quindi
il vincolo che non si può superare) e
l'ultima il valore della soluzione ottimale.

\begin{lstlisting}[caption={Rappresentazione sotto forma tabellare dell'istanza precedente}]
    ID  f0    f1    f2    f3    f4    f5   f6   f7   f8   f9   Value
    0   8.0   8.0   3.0   5.0   5.0   5.0  0.0  3.0  3.0  3.0   100.0
    1  12.0  12.0   6.0  10.0  13.0  13.0  0.0  0.0  2.0  2.0   600.0
    2  13.0  13.0   4.0   8.0   8.0   8.0  0.0  4.0  4.0  4.0  1200.0
    3  64.0  75.0  18.0  32.0  42.0  48.0  0.0  0.0  0.0  8.0  2400.0
    4  22.0  22.0   6.0   6.0   6.0   6.0  8.0  8.0  8.0  8.0   500.0
    5  41.0  41.0   4.0  12.0  20.0  20.0  0.0  0.0  4.0  4.0  2000.0
\end{lstlisting}
% \section{Nome del capitolo}
% \label{sec:capitolo}
% Aggiungi qui il contenuto della relazione !!\\
% \ \\
% Prova codice cmd
% \begin{lstlisting}[style=cmd]
%  this is a code in cmd style ...
% \end{lstlisting}
% \ \\
% Prova codice output
%
% \begin{lstlisting}[style=output]
%  this is a code in output style ...
% \end{lstlisting}
% \ \\
% Prova codice inline: \lstinline[style=cmd]|this is an inline code ...|\\
% \ \\