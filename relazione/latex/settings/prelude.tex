\usepackage[T1]{fontenc}
\usepackage[utf8]{inputenc}
\usepackage[italian]{babel}

\usepackage{graphicx}
%% links
\usepackage{hyperref}
% \hypersetup{
% 	colorlinks=true,
% 	linkcolor=blue,
% 	filecolor=magenta,
% 	urlcolor=cyan,
% 	pdftitle={\titolo},
% 	pdfpagemode=FullScreen,
% }

\graphicspath{ {./img/} }


%% configurazione per i listing di codice
\usepackage{float}
\usepackage{amsthm}
\usepackage{amsmath}
\usepackage{amssymb}
\usepackage{amsfonts}
\usepackage{xcolor}
\usepackage{listings}

\setlength{\parindent}{0pt}

\definecolor{codegreen}{rgb}{0,0.6,0}
\definecolor{codegray}{rgb}{0.5,0.5,0.5}
\definecolor{codepurple}{rgb}{0.58,0,0.82}
\definecolor{backcolour}{rgb}{0.95,0.95,0.92}

\lstdefinestyle{mystyle}{
	backgroundcolor=\color{backcolour},
	commentstyle=\color{codegreen},
	keywordstyle=\bfseries\color{black},
	numberstyle=\tiny\color{codegray},
	stringstyle=\color{codepurple},
	basicstyle=\ttfamily\footnotesize,
	breakatwhitespace=false,
	breaklines=true,
	captionpos=b,
	keepspaces=true,
	numbers=left,
	numbersep=5pt,
	showspaces=false,
	showstringspaces=false,
	showtabs=false,
	tabsize=2,
	mathescape=true,
	escapeinside={\%*}{*)},
	morekeywords={
			begin,
			end,
			if,
			then,
			else,
			to,
			endif,
			Procedure,
			while,
			do,
			return,
			true,
			false,
			set,
			for,
			repeat,
			until
			foreach},
}

\lstset{style=mystyle}

\renewcommand{\lstlistlistingname}{Elenco dei codici}
\renewcommand{\lstlistingname}{Codice}